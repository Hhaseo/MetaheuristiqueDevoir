\documentclass{beamer}

\usepackage[francais]{babel}
\usepackage[utf8]{inputenc}
\usepackage[T1]{fontenc}

\usepackage{beamerthemesplit}
\usetheme{Boadilla}
\usecolortheme{beaver}
\setbeamertemplate{navigation symbols}{}
\useinnertheme{circles}



\defbeamertemplate*{footline}{my infolines theme}
{
  \leavevmode%
  \hbox{%
  \begin{beamercolorbox}[wd=.333333\paperwidth,ht=2.25ex,dp=1ex,center]{author in head/foot}%
	\usebeamerfont{author in head/foot}\insertshortauthor~~\insertshortinstitute
  \end{beamercolorbox}%
  \begin{beamercolorbox}[wd=.333333\paperwidth,ht=2.25ex,dp=1ex,center]{title in head/foot}%
	\usebeamerfont{title in head/foot}\insertshorttitle
  \end{beamercolorbox}%
  \begin{beamercolorbox}[wd=.333333\paperwidth,ht=2.25ex,dp=1ex,right]{date in head/foot}%
	\usebeamerfont{date in head/foot}\insertshortdate{}\hspace*{2em}
	\insertframenumber{} / \inserttotalframenumber\hspace*{2ex}
  \end{beamercolorbox}}%
  \vskip0pt%
}



\title[Métaheuristique : Algorithme BVNS]
{%
	Métaheuristique : Algorithme BVNS
}
\author{Guillaume~\textsc{Desquesnes} \& Florian~\textsc{Benavent}}
\date{13 février 2015}

\AtBeginSection[]
{
	\begin{frame}{Plan}
		\begin{columns}
			\column{0.2\textwidth}
			\column{0.8\textwidth}
				\tableofcontents[currentsection]
		\end{columns}
	\end{frame}
}
\begin{document}

	\frame{\titlepage}

	\section{Présentation}

		\begin{frame}{Basic Variable Neighborhood Search}
			\begin{itemize}
				\item Fonction de voisinage est très importante pour la recherche globale
				\item Utilisation plusieurs fonctions de voisinage pour améliorer la recherche
			\end{itemize}
		\end{frame}
		
		
		\begin{frame}{Incop}
			\begin{itemize}
				\item Librairie C++ de résolution de problème d'optimisation
				\item Possède des méthodes de recherche locale
				\item Peut lire les instances DIMACS
			\end{itemize}
		\end{frame}
	
	\section{Structures de voisinage}
		\begin{frame}{Structures de voisinage}
			\begin{description}
				\item[p-Flip]On modifie au plus la valeur de p variable en conflit par une autre valeur aléatoire de leur domaine.

				\item[swap]\'Echange les valeurs d'une variable en conflit et de la variable la précédent ou la suivant dans la représentation des solutions.

				\item[2-Exchange]\'Echange les valeurs d'une variable en conflit et d'une autre choisit aléatoirement parmis les variable restantes.

				\item[Kempe chain]À partir d'une arête possédant une variable en conflit, on construit la composante connexe des variables de valeur identique à un des
				noeuds de l'arête. On échange ensuite les valeurs des variable de la composante connexe.
			\end{description}
		\end{frame}
	
	\section{Intégration dans Incop}
		\begin{frame}{Structure de voisinage}
			\begin{itemize}
				\item Fait le shaking
				\item Retourne une nouvelle solution (\texttt{Configuration}) du problème
			\end{itemize}
		\end{frame}
		
		\begin{frame}{BVNS}
			\begin{itemize}
				\item Surcharge d'\texttt{IncompleteAlgorithm}
				\item Utilise une recherche locale d'Incop spécifiée en argument
			\end{itemize}
		\end{frame}
		
	\section{Demo}
	
	\section{Résultats}
	
		\begin{frame}{Description}
			\begin{itemize}
				\item 4 recherche de solution par instance
				\item Recherche locale taboue
				\item Voisinage pFlip puis Swap puis 2-Exchange puis Kempe chain
				\item Arrêt de recherche au bout de 30 secondes ou de solution.
				\item Affiche le nombre d'arcs en conflit
			\end{itemize}
		\end{frame}
		
		\begin{frame}{Fichier DSJC}
	\begin{tabular}{|l|c|c|c|c|}
  	\hline
	Nombre de couleurs & 5 & 6 & 7 & 9 \\
  	\hline
	DSJC125.5 & 300 & 234 & 188 & 129 \\
	DSJC250.5 & 1 344 & 1 073 & 887 & 630 \\
	DSJC500.5 & 5 716 & 4 640 & 3 884 & 2 894 \\
	DSJC1000.5 & 23 571 & 19 309 & 16 313 & 12 344 \\
	\hline
	\end{tabular}
			
		\end{frame}
		
		\begin{frame}{Fichier Flat}
		
	\begin{tabular}{|l|c|c|c|c|}
  	\hline
	Nombre de couleurs & 28 & 29 & 30 & 32 \\
  	\hline
	flat300\_28\_0 & 155 & 135 & 98 & 117 \\
	\hline
	\end{tabular}
	
	\begin{tabular}{|l|c|c|c|c|}
  	\hline
	Nombre de couleurs & 50 & 51 & 52 & 54 \\
  	\hline
	Flat1000\_50\_0 & 1458 & 1400 & 1390 & 1328 \\
	\hline
	\end{tabular}
	
	\begin{tabular}{|l|c|c|c|c|}
  	\hline
	Nombre de couleurs & 60 & 61 & 62 & 64 \\
  	\hline
	Flat1000\_60\_0 & 1171 & 1139 & 1089 & 1087 \\
	\hline
	\end{tabular}

		\end{frame}
		
		\begin{frame}{Fichier Le}
			\begin{tabular}{|l|c|c|c|c|}
  	\hline
	Nombre de couleurs & 25 & 26 & 27 & 29 \\
  	\hline
le450\_25a & 0 (9.42s) & 0 (0.99s) & 0 (0.27s) & 0 (0,12s) \\
le450\_25b & 0 (3.75s) & 0 (0.43s) & 0 (0.13s) & 0 (0.08s) \\
le450\_25c & 84 & 68 & 57 & 31 \\
le450\_25d & 89 & 75 & 62 & 34 \\
	\hline
		\end{tabular}


		\end{frame}
	
\end{document}
